\documentclass[svgnames]{article}   	% use "amsart" instead of "article" for AMSLaTeX format
%\geometry{landscape}                	% Activate for rotated page geometry

%\usepackage[parfill]{parskip}    		% Activate to begin paragraphs with an empty line rather than an indent

\usepackage{graphicx}				          % Use pdf, png, jpg, or eps§ with pdflatex; use eps in DVI mode
\setcounter{secnumdepth}{4}
\setcounter{tocdepth}{4}
%maths							                  % TeX will automatically convert eps --> pdf in pdflatex		
\usepackage{amssymb}
\usepackage{amsmath}
\usepackage{esint}
\usepackage{geometry}

%pgfplots
\usepackage{pgfplots}

%images
\graphicspath{{/Users/devaldeliwala}}					          % Activate to set a image directory 

%tikz
\usepackage{pgfplots}
\pgfplotsset{compat=1.15}
\usepackage{comment}
\usetikzlibrary{arrows}
\usepackage[most]{tcolorbox}

%Figures
\usepackage{float}
\usepackage{caption}
\usepackage{lipsum}


\title{5B - Introductory Electromagnetism, Waves, and Optics}
\author{deval deliwala}
%\date{}							                % Activate to display a given date or no date

\begin{document}
\maketitle
%\section{}
%\subsection{}
\tableofcontents 					           % Activate to display a table of contents
\newpage

\noindent \textbf{Jan 17} 
\hrule
\section{Maxwell's Equations}

\begin{align*} 
      &\nabla \cdot \vec{E} = \frac{\rho}{\varepsilon_0}      \\  
      &\nabla \times \vec{B} = \mu_0\vec{J} + \mu_0\varepsilon_0 \frac{\partial
      \vec{E}}{\partial t} \\
      &\nabla \times \vec{E} = - \frac{\partial \vec{B}}{\partial t} \\
      &\nabla \cdot \vec{B} = 0
\end{align*}
\vspace{5px}

The goal of this course will be to understand Maxwell's Equations, and the
unison between the electric and magnetic field. 
\vspace{5px}

\begin{tcolorbox}[colback = red!5!white, colframe = red!50!black, title
  = Lorent'z Force eq]
  \[
  \vec{F} = q\vec{E} + q\vec{v} \times \vec{B}
  \]
\end{tcolorbox}
\vspace{5px}
\begin{tcolorbox}[colback = blue!5!white, colframe = blue!50!black, title
  = Rules]
  
  1. Charge in nature is quantized in units of $e$ \\
  2. Charge is conserved \\
  3. Charge has 2 types: $\pm$

\end{tcolorbox}
\vspace{5px}

$\vec{E}$ and $\vec{B}$ are \textit{vector fields}. This means $\vec{E}$ is
a function of every point in space: $\vec{E}(x,y,z)$

\vspace{5px}\[
  \vec{E}(r) = E_x(x,y,z)\hat{i} + E_y(x,y,z)\hat{j} + E_z(x,y,z)\hat{k}
\] \vspace{5px}
\newpage
$\nabla$ is a vector operator: 

\vspace{5px} \[
  \frac{\partial }{\partial x} \hat{i} + \frac{\partial }{\partial y} \hat{j}
  + \frac{\partial }{\partial x} \hat{k}
\] \vspace{5px}

\begin{tcolorbox}[colback = red!5!white, colframe = red!50!black, title
  = Vector Operations]
  
  \[
    \vec{A}\alpha \rightarrow \vec{B}
  \]
  \[
    \vec{A} \cdot \vec{B} \rightarrow \alpha
  \]
  \[
    \vec{A} \times \vec{B} \rightarrow \vec{C}
  \]

\end{tcolorbox}
\vspace{5px}

Consider a scalar field: $\varphi(\vec{r})$
\begin{align*}
&d\varphi = \varphi(r + dr) - \varphi(r) \\
&d\varphi = \frac{\partial \varphi(x,y,z)}{\partial x}dx + \frac{\partial
\varphi(x,y,z)}{\partial y}dy + \frac{\partial \varphi(x,y,z)}{\partial z}dz\\
&d\vec{r} = \hat{i}dx + \hat{y}dy + \hat{z}dz
\end{align*}
\begin{tcolorbox}
\[
  d\varphi = \nabla \varphi(\vec{r}) \cdot d\vec{r}
\]
\end{tcolorbox}
\vspace{5px}

$\nabla \varphi$ = "gradient of $\varphi$. " Also $\nabla \cdot \vec{E}
= \frac{\partial E_x}{\partial x}+ \frac{\partial E_y}{\partial y}
+ \frac{\partial E_z}{\partial z}$. This is known as the \textbf{divergence} of
$\varphi$. 
\vspace{10px}

The \textbf{curl} of $\varphi$ is equal to 

\vspace{5px} \[
\begin{vmatrix}
  \hat{i} & \hat{j} & \hat{k} \\
  P & Q & R \\
  \frac{\partial }{\partial x} & \frac{\partial }{\partial y} & \frac{\partial
  }{\partial z}  
\end{vmatrix}
\] \vspace{5px}
\[
  \rho = \text{charge density} = \frac{\text{number of particles} q}{dV} = nq
\] \vspace{5px}
where $n = \frac{\text{number of particles}}{dv}$


\vspace{5px} \[
  \vec{J} = \text{current density}
\] \vspace{5px}

\section{Statics}

\begin{tcolorbox}[colback = red!5!white, colframe = red!50!black, title
  = Equations of Electrostatics]
  
  \[
    \nabla \cdot \vec{E} = \frac{\rho}{\varepsilon_0}
  \]
  \[
    \nabla \times \vec{E} = 0
  \] 
  
\end{tcolorbox}

\begin{tcolorbox}[colback = blue!5!white, colframe = blue!50!black, title
  = Equations of Magnetostatics]
  
  \[
    \nabla \cdot \vec{B} = 0
  \]
  \[
    \nabla \times \vec{B} = \mu_0\vec{J}
  \] 

\end{tcolorbox}

\subsection{Flux}
\vspace{5px}
"Flux" = Flow 

\vspace{10px} 

Consider a fluid flow with a velocity vector field $\vec{v}(\vec{r})i$, flowing
into a small aperture defined by $\hat{n}\,d\vec{a}$, where  $\hat{n}$ is the
unit normal vector to the aperture, and $d\vec{a}$ is the area. 

\[
  d\Phi = \vec{v} \cdot d\vec{a}
\]

Relating to the Electric field, 

\[
  d\Phi_E = \vec{E}(\vec{r}) \cdot d\vec{a}
\] 
\[
  \int_S d\Phi_E = \Phi_E = \int_S \vec{E}(\vec{r}) \cdot d\vec{a}
\]\vspace{5px}

Green's, Gauss's, Divergence Theorem

\[
  \int_S \vec{E} \cdot d\vec{a} = \iiint_S \nabla \cdot \vec{E}(\vec{r}) \, d^3
  r
\]

\newpage
\noindent \textbf{Jan 19} \hrule
\vspace{5px} \[
  \Phi = \oint_S \vec{E}(\vec{r}) \cdot d\vec{a} = \iiint_V d^3r \, \nabla \cdot
  \vec{E}(\vec{r}) \rightarrow \text{Divergence Theorem}
\] \vspace{5px}


%1
% flux from two sides of a cube

\begin{figure}[htb!]
  \centering
    \includegraphics[width = 10cm]{screenshot 28.png}
    \caption{Flux from two sides of cube}
\end{figure}
\vspace{5px}


\begin{align*}
  d\Phi_{12} &= \vec{E}(x,y,z) \cdot d\vec{a_1} + \vec{E}(x+dx, y,z) \cdot
  d\vec{a_2} \hspace{10px} | \hspace{10px} 
   d\vec{a_1} = -\hat{i}dydz, d\vec{a_2} = \hat{i}dydz\\ \\
  d\Phi_{12} &=[ E_x (x + dx, y,z) -  E_x(x,y,z) ] \, dydz\\
  d\Phi_{12} &= \left[\frac{\partial E_x(x,y,z)}{\partial x} \, dx\right]
  dydz\\ \\
\text{Since } dv = dxdydz, \\ \\
  d\Phi_{12} &= \frac{\partial E_x(x,y,z)}{\partial x}  \cdot d^3v\\
  d\Phi_{tot} &= d\Phi_{12} + d\Phi_{34} + \dots\\
              &= \left( \frac{\partial E_x}{\partial x}  + \frac{\partial E_y}{\partial y}
  + \frac{\partial E_z}{\partial z} \right) \, d^3 r \\
  d\Phi_{tot} &= \nabla \cdot \vec{E}(\vec{r}) \, d^3 r\\\\
  \nabla \cdot \vec{E} &= \frac{d \Phi}{d^3 r} \\
  \nabla \cdot E &= \frac{\text{flux}}{\text{volume}}
\end{align*} 
\vspace{5px}

%2
% flux from the boundary of the surface is equivalent to adding the flux of
% every $dV$ insidei

\begin{figure}[htb!]
  \centering
    \includegraphics[width = 10cm]{screenshot 29.png}
    \caption{Flux from the boundary surface is equal to adding the flux of
    every $dV$ inside}
\end{figure}




\subsection{Divergence Theorem}

\begin{tcolorbox}	
  \[ 
  \iiint \nabla \cdot \vec{E} (\vec{r} ) \, d^3r = \oint_S \vec{E} \cdot d\vec{a}
  \]
\end{tcolorbox}	

\vspace{5px} \[
\iiint \frac{\rho(\vec{r})}{\epsilon_0} d^3r = \oint \vec{E}(\vec{r} ) \cdot
d\vec{a}
\] \vspace{5px}


\begin{tcolorbox}[colback = red!5!white, colframe = red!50!black, title = Gauss' Law]
  \[
    \Phi = \frac{Q}{\varepsilon_0}
  \]
\end{tcolorbox}

%3
% flux from positive charge is positive (source), flux from negative charge is
% negative (sink)

\begin{figure}[H]
  \centering
    \includegraphics[width = 8cm]{screenshot 30.png}
    \caption{Flux from positive charge is positive (source), flux from negative
    charge is negative (sink)}
\end{figure}


\subsection{Coulomb's Law}
\vspace{5px}

%4
% electric field from a sphere is directly outward and has same magnitude due to symmetry

\begin{figure}[htb!]
  \centering
    \includegraphics[width = 8cm]{screenshot 31.png}
    \caption{Electric field from a sphere is directly outward and has the same
    magnitude due to symmetry}
\end{figure}

\subsection{Electric Field of Point Charge}
\vspace{5px}
\begin{align*}
\vec{E} (\vec{r} ) \propto \hat{r} \\
\Phi &= \frac{Q}{\epsilon_0}\\
E(r) \cdot 4\pi r^2 &= \frac{Q}{\epsilon_0}\\
\vec{E}(\vec{r} ) &= \hat{r}\frac{Q}{4\pi \epsilon_0 r^2}
\end{align*}

%5
%diagram of coulomb's law
\begin{figure}[H]
  \centering
    \includegraphics[width = \linewidth]{screenshot 32.png}
    \caption{Diagram of Coulomb's Law}
\end{figure}



\begin{tcolorbox}[title = Coulomb's Law]
\[
  \vec{F_{21}} = q_2 \vec{E_1}(P_2) = \frac{q_2q_1}{4 \pi \epsilon_0 r^2}
  \hat{r_{21}}
\]
\end{tcolorbox}

The force is proportional to the product of the charges, and inversely
proportional to the square of the distance, and directed along the line between
the charge, and directed along the line between the charges. Attractive for
opposite signs of charge, and vice versa. 

%6
%comparison between $F_g$ and $F_c$ i

\begin{figure}[H]
  \centering
    \includegraphics[width = 8cm]{screenshot 33.png}
    \caption{Comparison between $F_g$ and $F_c$}
\end{figure} 

\newpage
\textit{ \textbf{Jan 19 Summary}} 
\vspace{5px} \[
  \iiint_V \nabla \cdot \vec{E} (\vec{r} ) \, dxdydz = \oint_S \vec{E} \cdot
  d\vec{a}
\] \vspace{5px}
\vspace{5px} \[
  \Phi = \frac{Q}{\epsilon_0} \rightarrow \vec{E}_{\text{point charge}}
  = \frac{Q}{4\pi r^2 \epsilon_0}
\] \vspace{5px}
\newpage
\noindent \textbf{Jan 24} \hrule
\vspace{10px} 

\begin{tcolorbox}[colback = red!5!white, colframe = red!50!black, title
  = Calculating $\vec{E}$ for arbitrary $\rho(\vec{r)}$]
  
  Superposition: $\nabla \cdot \vec{E} = \frac{\rho}{\epsilon_0}$, works
  because Maxwell's Equation are linear.

  \begin{align*}
    &\nabla \cdot \vec{E_1} = \frac{\rho_1}{\varepsilon_0} \\
    &\nabla \cdot \vec{E_2} = \frac{\rho_2}{\varepsilon_0} \\\\
    &\nabla \cdot (\vec{E_1} + \vec{E_2}) = \frac{\rho_1 + \rho_2}{\varepsilon_0}
  \end{align*}

  Proved $\nabla \cdot \vec{E_T} = \frac{\rho_T}{\varepsilon_0}$
\end{tcolorbox}

%1
%volumes of charge
\begin{figure}[htb!]
  \centering
    \includegraphics[width = 7cm]{screenshot 41.png}
    \caption{volumes of charge}
\end{figure}



%2
%system of charges, notation of vectors


\begin{figure}[htb!]
  \centering
    \includegraphics[width = 5cm]{screenshot 42.png}
    \caption{system of charges}
\end{figure}

\begin{figure}[htb!]
  \centering
    \includegraphics[width = 8cm]{screenshot 43.png}
    \caption{notations of vectors, $\vec{r}, \vec{r_1}, \vec{r} - \vec{r_1}$}
\end{figure}


\begin{tcolorbox}[colback = blue!5!white, colframe = blue!50!black, title
  = System of point charges]

Therefore the field at $P$ is

\[
\vec{E_1}(\vec{r}) = \frac{q_1}{4\pi\varepsilon_0}\frac{1}{|\vec{r}
- \vec{r_1}|^2} \cdot \frac{(\vec{r} - \vec{r_1})}{|\vec{r}- \vec{r_1}|}
= \frac{q_1}{4\pi\varepsilon} \frac{(\vec{r}-\vec{r_1})}{|\vec{r}  - \vec{r_1}
|^3}
\]
Superposition: 

\[
\vec{E}(\vec{r} ) = \sum_i \frac{q_i}{4\pi\varepsilon_0} \frac{(\vec{r}
- \vec{r_1} )}{|\vec{r} - \vec{r_1} |^3}
\]
\end{tcolorbox}

%3
%volume charge distribution, $dq = \rho(r) d^3 v$

\begin{figure}[htb!]
  \centering
    \includegraphics[width = 8cm]{screenshot 44.png}
    \caption{volume charge distribution, $dq = \rho(r) d^3 v$}
\end{figure}

\begin{figure}[htb!]
  \centering
    \includegraphics[width = 3cm]{screenshot 45.png}
    \caption{using symmetry to determine component of $\vec{E}$}
\end{figure}



\begin{tcolorbox}	
  
  \[
  \vec{E}(\vec{r}) = \frac{1}{4\pi\varepsilon_0} \int d^3r'
  \frac{\rho(\vec{r}')(\vec{r} - \vec{r}' }{|\vec{r} - \vec{r}'|^3}
  \]
  \[
    \nabla \cdot \vec{E} = \frac{\rho}{\varepsilon_0}
  \]

\end{tcolorbox}	


\subsection{Gauss's Law + Symmetry} 

%4
%different Gauss' law symmetries, symmetry of sphere of charge

\begin{figure}[H]
  \centering
    \includegraphics[width = 10cm]{screenshot 46.png}
    \caption{different Gauss's Law symmetries, symmetry of a sphere of charge}
\end{figure}


\paragraph{Symmetry of cylinder of charge}

%5
%symmetry of cylinder - field has to be radial

\begin{figure}[H]
  \centering
    \includegraphics[width = 10cm]{screenshot 47.png}
    \caption{symmetry of a cylinder of charge}
\end{figure}


\paragraph{Symmetry of plane of charge}

%6
%symmetry of plane - mirror symmetry

\begin{figure}[H]
  \centering
    \includegraphics[width = 10cm]{screenshot 48.png}
    \caption{symmetry of a plane -- mirror symmetry}
\end{figure}



\paragraph{Using Gauss's Law symmetry}

\textbf{Problem. Uniform Sphere of Charge}  
  
\begin{figure}[H]
  \centering
    \includegraphics[width = 7cm]{screenshot 49.png}
    \caption{uniform sphere of charge, inside and outside the radius}
\end{figure}



\begin{align*}
  \text{Gauss's Law} \\\\
  &\oint_S = \vec{E} \cdot d\vec{a} = \frac{Q}{\varepsilon_0}\\\\
  \text{Outside Surface}\\
  &E(r) \cdot 4\pi r^2 = \frac{\rho_0 \frac{4\pi
  r^3}{3}}{\varepsilon_0 } \\
  &E(r) = \frac{\rho_0 R^3}{3\varepsilon_0 r^2} = \frac{Q}{4\pi r^2
  \varepsilon_0} \\\\
  \text{Inside Surface} \\
  &E(r) \cdot 4\pi r^2 = \frac{Q}{\varepsilon_0} \cdot \left(\frac{r}{R}\right)^3
\end{align*}

\begin{tcolorbox}	
    Outside: 
    \[E(r) = \frac{Q}{4\pi r^2 \varepsilon} \]
  Inside: 
  \[E(r) = \frac{Q}{4\pi \varepsilon} \frac{r}{R^3} \]
\end{tcolorbox}	


%7
%graph of $E(r) v. $r$ for a spherical charge distribution

\begin{figure}[H]
  \centering
    \includegraphics[width = 8cm]{screenshot 50.png}
    \caption{graph of $E(r)$ v.  $r$ for a spherical charge distribution}
\end{figure}





\paragraph{Lower Dimensional Charge Distributions}

%8
%1 and 2 dimensional charge distributions - $\sigma$ and $\lambda$

\begin{figure}[H]
  \centering
    \includegraphics[width = 6cm]{screenshot 51.png}
    \caption{1 and 2 dimensional charge distributions - $\sigma$ and $\lambda$}
\end{figure}




\paragraph{Applying Gauss's Law to a Plane of Charge}

%9
%gaussian surface of a plane of charge

\begin{figure}[H]
  \centering
    \includegraphics[width = 8cm]{screenshot 52.png}
    \caption{gaussian surface of a plane of charge}
\end{figure}



\begin{align*}
  2E(r)A &= \frac{\sigma A}{\varepsilon_0} \\
  E &= \frac{\sigma}{\varepsilon_0} 
\end{align*}

\end{document}  

