\documentclass[svgnames]{article}   	% use "amsart" instead of "article" for AMSLaTeX format
%\geometry{landscape}                	% Activate for rotated page geometry

%\usepackage[parfill]{parskip}    		% Activate to begin paragraphs with an empty line rather than an indent

\usepackage{graphicx}				          % Use pdf, png, jpg, or eps§ with pdflatex; use eps in DVI mode
\setcounter{secnumdepth}{4}
\setcounter{tocdepth}{4}

%maths							                  % TeX will automatically convert eps --> pdf in pdflatex		
\usepackage{amssymb}
\usepackage{amsmath}
\usepackage{esint}
\usepackage{geometry}

%pgfplots
\usepackage{pgfplots}

%images
\graphicspath{{/Users/devaldeliwala}}					          % Activate to set a image directory 

%tikz
\usepackage{pgfplots}
\pgfplotsset{compat=1.15}
\usepackage{comment}
\usetikzlibrary{arrows}
\usepackage[most]{tcolorbox}

%Figures
\usepackage{float}
\usepackage{caption}
\usepackage{lipsum}


\title{Physics 89 - Introduction to Mathematical Physics}
\author{deval deliwala}
%\date{}							                % Activate to display a given date or no date

\begin{document}
\maketitle
%\section{}
%\subsection{}
\tableofcontents 					           % Activate to display a table of contents
\newpage

\noindent \textbf{Jan 17} \hrule


\section{Difference between Mathematics and Physics}
\vspace{5px}

\begin{tcolorbox}[title = Example 1 - Electrostatics]

\textbf{Math Question}
\[
  x + \frac{x^2}{2} + \frac{x^3}{3} + \frac{x^4}{4} + \dots = ? 
\]

\noindent \textbf{Math Solution}
\[
x + \frac{x^2}{2} + \frac{x^3}{3} + \dots = -\log(1-x), \hspace{10px} for -1
\leq x \leq 1
\]

So, 
\[
-1 + \frac{1}{2} - \frac{1}{3} + \frac{1}{4} - \frac{1}{5} + \dots = -\log(2)
\]

\end{tcolorbox}
\vspace{5px}
\begin{tcolorbox}[title = Example 2 - Diffusion] 
  
 \[
   f(x,y,z,t) = \text{density of diffusing material at time $t$}
 \]
 Let there exist a cube containing moles 

 \[
 \frac{\partial f}{\partial t} = D\left( \frac{\partial^2 f}{\partial x^2}
 + \frac{\partial^2 f}{\partial y^2} + \frac{\partial^2 f}{\partial z^2}\right)
 \]
 where $D$ is the \textit{diffusion coefficient}, and the diffusion equation
 describes how $f$ evolves with time \\

 \textbf{Math Question} \\

 Solve 

 \[
 \frac{\partial f}{\partial t} = D \left( \frac{\partial^2 f}{\partial x^2}
   + \frac{\partial^2 f}{\partial y^2} + \frac{\partial^2 f}{\partial
   z^2}\right)
 \]
 given initial condition 

  \[
    f(x,y,z,0) = \text{concentrated lump at the origin} 
 \]
 
 \textbf{Math Solution}
 
\[
  f(x,y,z,t) = \frac{N}{(4\pi D t)^(3/2)}e^{-\frac{x^2 + y^2 + z^2}{4Dt}}
\] \vspace{5px}

where $N$ is the number of moles released
  

\end{tcolorbox}	

\newpage
\noindent \textbf{Jan 19} \hrule
\section{Taylor Series} 

\begin{itemize}
  \item Techniques for obtaining series
  \item Estimate error, converge?     
\end{itemize}
%1

\begin{figure}[htb!]
  \centering
    \includegraphics[width = 7cm]{screenshot 35.png}
    \caption{Taylor Series Visualization}
\end{figure}


\vspace{5px} \[
  f(x) \approx f(0) + f'(0)x + \dots + \frac{1}{n!}f^{n}(0)x^n
\] \vspace{5px} 
\[
  f(x) \approx f(a) + f'(a)(x-a) + \frac{1}{2}f''(a)(x-a)^2 + ... = \sum_{k=0}^{\infty} \frac{1}{k!}f^{k} (a)
  (x-a)^k
\]

\begin{tcolorbox}[colback = red!5!white, colframe = red!50!black, title
  = Question]
  
  How good is this approximation? 

\end{tcolorbox}

\textit{ Big $O$ notation}
\vspace{5px} \[
\sum_{k=9}^{n} \frac{1}{k!}f^k(0) x^k + O(x^{n+1})
\] \vspace{5px}

\textit{ \textbf{Formally,}}

\vspace{5px} \[
  F(x) = o(x^{n+1}) \hspace{10px} \text{as } x \rightarrow 0
\] \vspace{5px}
\[
  |F| \leq C|x|^{n+1} \hspace{10px} \text{for some unexpected constant c}
\] \vspace{5px}
\[
  \lim_{x\to 0} \frac{F}{|x|^{n+1}} = 0
\] \vspace{5px}
\begin{tcolorbox}[colback = blue!5!white, colframe = blue!50!black, title
  = Example]
  \begin{gather*}
  e \approx 1.9 GeV \approx 3700 mc^2 \\\\
  \text{Special Relativity} \\
  E_k = m_0 c^2 - mc^2 = \frac{mc^2}{ \sqrt{ 1 - \frac{v^2}{c^2}}} - mc^2 \\
  \approx 0 + \frac{1}{2}mv^2 + \frac{3}{8}m\frac{v^4}{c^2} + \frac{5}{16}m
  \frac{v^8}{c^4} \\
  f(v) = \frac{1}{2}mv^2  + \frac{3}{8}m\frac{v^4}{c^2} + \dots 
  \end{gather*}
\end{tcolorbox}

\vspace{5px} \[
\frac{1}{ \sqrt{1-x}} \rightarrow \frac{mc^2}{ \sqrt{1 - \frac{v^2}{c^2}}}
\] \vspace{5px}
\[
  (1+x)^P, \hspace{10px} \text{then set $p = \frac{1}{2}$}
\]

\begin{align*}
  f(x) &= (1+x)^n  \\ f'(x) &= p(1+x)^{p-1}  \\ f^k (x)  &= p(p-1) \dots
  (p-k+1)(1+x)^{p-k} \rightarrow f^k (0) \\ &= p \dots (p-k+1) 
\end{align*}

\begin{align*}
  (1+x)^n \approx 1 + px + \frac{p(p-1)}{2!}x^2 + ... + \frac{p!}{k!(p-k)!}x^k
  = \begin{pmatrix} p \\ k \end{pmatrix}x^k
\end{align*}

\vspace{5px} \[
  \sum_{k=0}^{n} \begin{pmatrix} p \\ k \end{pmatrix} x^k \hspace{10px}
  \text{generalized binomial coefficient} 
\] \vspace{5px}
\vspace{5px} \[
  (1+x)^P = \sum_{k=0}^{n} \begin{pmatrix} p \\ k \end{pmatrix} x^k + O(x^{n+1}
\] \vspace{5px}

\begin{tcolorbox}[colback = red!5!white, colframe = red!50!black, title
  = Question]
  
  Given $\frac{1}{ \sqrt{1+x}}$ taylor series, how good is this approximation
  if $x=0.1$? 

\end{tcolorbox}
\begin{tcolorbox}[colback = blue!5!white, colframe = blue!50!black, title
  = Solution]
\vspace{5px} \[
  \text{Actual Answer} \rightarrow  \frac{1}{\sqrt{1.1}} = 0.9534626
\] \vspace{5px}
\[
  \text{Taylor Polynomials $x, x^2$} \rightarrow 1 - \frac{0.1}{2} = 0.95
  \hspace{10px} / \hspace{10px} 1 - \frac{0.5}{2} + \frac{3(0.5)^2}{8}
  = 0.95375 \hspace{10px} \text{good approx}
\] \end{tcolorbox}
\textit{ \textbf{More Taylor Series}}

\begin{align*}
  &\sin x = x - \frac{x^3}{3!} + \frac{x^5}{5!} - \frac{x^7}{7!} + \dots \\ 
  &\cos x = 1 - \frac{x^2}{2!} + \frac{x^4}{4!} - \frac{x^6}{6!} + \dots \\ 
  &e^x = 1 + x + \frac{x^2}{2!} + \frac{x^3}{3!} + \frac{x^4}{4!} + \dots \\\\
  &\cosh x = \frac{e^x + e^{-x}}{2} = 1 + \frac{x^2}{2!} + \frac{x^4}{4!}
  + \frac{x^6}{6!} + \dots \\
  &\sinh x = \frac{e^x - e^x}{2} = x - \frac{x^3}{3!} + \frac{x^5}{5!} + \dots \\ 
  %&J_o (2x) = \sum_{k=0}^{\infty} \frac{x^{2k}}{(k!)^2}(-1)^k
  %= 1 - \frac{x^2}{1!^2 + \frac{x^4}{2!^2} - \frac{x^6}{3!^2} + \dots = \text{ Bessel Function}
 \end{align*}


\subsection{Testing for Convergence}

If $ \sum_{0}^{\infty} a_n x^n \leq \infty $ converges, 

\vspace{5px} \[
\sum_{0}^{\infty} a_n (\lambda X)^n \leq \infty \hspace{10px}  \hspace{10px}
|\lambda| \leq 1
\] \vspace{5px}

Taylor Series have interval of convergence of the form 

\vspace{5px} \[
  [-L, L] \hspace{10px} (-L, L) \hspace{10px} [-L, L) \hspace{10px} (-L, L]
\] \vspace{5px}

\begin{tcolorbox}[colback = red!5!white, colframe = red!50!black, title
  = Truncated Taylor Series Approximation]
  
 \vspace{5px} \[
 R_0(x) = f(x) - f(0) = f'(c)x
 \] \vspace{5px} 

\end{tcolorbox}

%4
%remainder visualized

\begin{figure}[htb!]
  \centering
    \includegraphics[width = 10cm]{screenshot 38.png}
    \caption{Remainder Visualized}
\end{figure}



\begin{tcolorbox}[colback = blue!5!white, colframe = blue!50!black, title
  = Remainder Theorem]
 \[ 
  R_n(x) = f^{n+1} (c) \frac{x^{n+1}}{(n+1)!} \hspace{10px} \text{for some $0
  \leq c \leq x$}
\]
\end{tcolorbox}

\begin{tcolorbox}	
  
  \begin{align*}
    x &= \frac{\pi}{2} \\ 
    R &= \sin \frac{\pi}{2} - (x - \frac{x^3}{6} + \frac{x^5}{120}
    - \frac{x^7}{5040} + \frac{x^9}{362880..} + 0 ) \\
      &= f^{10}(c)\frac{x^10}{10!} \hspace{10px} 0 \leq c \leq \frac{\pi}{2}
      \\\\
    |f^{11} (c) | &= | -\cos c| < 1 \\
    |R_{10}| &\leq \frac{1}{11!} \left(\frac{\pi}{2}\right)^{11} \approx 3.6
    \times 10^{-6}
  \end{align*}

\end{tcolorbox}	
\newpage
\textbf{ \textit{Technique for Solving Taylor Series by dividing two
polynomials}}
\begin{align*}
f(x) &= a_0 + a_1x + \dots  \\
g(x) &= b_0 + b_1x + \dots \\
\frac{f(x)}{g(x)} &= (c_0 + c_1x +c_2x^2 + \dots) \\
a_0 + a_1x + \dots &= (b_0 + b_1x + \dots)(c_0 + c_1x + \dots)  \\\\
a_0 &= b_0c_0
\end{align*}

\newpage
\noindent \textbf{Jan 24} \hrule
\vspace{10px} 
\section{Complex Numbers}

\begin{itemize}
  \item Definition
  \item Functions: $\log z, \sqrt{z}, \sin z,$, etc.
  \item Applications: AC Circuits, Hydrodynamics
  \item Math Applications: $\int_{\infty}^{\infty}$
\end{itemize}

\subsection{Taylor Series}

\[
f(x) = \frac{1}{1+x^2} = \frac{1}{1-(-x^2)} = 1 - x^2 + x^4 - x^6 + - \dots
= \sum_{n=0}^{\infty} (-1)^n x^{2n}
\]

The interval of convergence for the taylor series of $\frac{1}{1+x^2}$ is from
$(-1,1)$, which is not readily apparent since 

 \[
@x \pm 1, f(x) = \frac{1}{2}
\]

%1
%taylor series of $e^{1/x^2}$

\begin{figure}[H]
  \centering
    \includegraphics[width = 7cm]{screenshot 53.png}
    \caption{taylor series of $e^{1/x^2}$}
\end{figure}



\subsection{Complex Numbers}

Introduced by \textit{Cardano} in the 1500s with the intent of solving cubic
equations.

\begin{tcolorbox}[colback = red!5!white, colframe = red!50!black, title
  = Quadratic Equations]
  
  \[
  0 = x^2 + bx + c \hspace{10px} x = -b \pm \frac{ \sqrt{b^2 - 4ac}}{2a}
  \]
\end{tcolorbox}

\begin{tcolorbox}[colback = blue!5!white, colframe = blue!50!black, title
  = Cubic Equations]
  \[
  0 = x^3 + ax + b \hspace{10px} \left(\frac{-b}{2} + \sqrt{\frac{b^2}{4}
  - \frac{a^3}{27}}\right)^{\frac{1}{3}}
  \]

  \[
    x^3 - x = 0 \rightarrow x = \frac{1}{ \sqrt{3}}\left[\sqrt{-1}^{1/3}
    + (-\sqrt{-1})^{1/3} \right]
  \]
  \begin{itemize}
    \item consistency
    \item final answer is \textbf{real}
    \item simplifies computations
  \end{itemize}  
  
\end{tcolorbox}

\paragraph{Rules of Complex Numbers}

\[
z = a + bi
\]

\[
 i^2 = -1
\]

\[
  (a + bi)(c + di) = (ac - bd) + (ad + bc)i
\]
\vspace{5px}
\begin{tcolorbox}[title = Example]

  \begin{align*}
    &(1+i)^2 = 2i \\
    &i^4 = 1
  \end{align*}

\end{tcolorbox}
\vspace{5px}

\begin{align*}
  \frac{1}{a+bi} &= \frac{(a+bi)}{(a-bi)(a+bi)} = \frac{(a-bi)}{a^2 + b^2} \\
                 &= \left(\frac{a}{a^2+b^2}\right) - \left(\frac{b}{a^2 + b^3}\right)i
\end{align*}

\subsection{Applications}

\paragraph{Hydrodynamics}

%2
%2D Diagram of Sphere from above 

\begin{figure}[H]
  \centering
    \includegraphics[width = 9cm]{screenshot 54.png}
    \caption{2D diagram of Sphere from above}
\end{figure}


\[
\vec{v}(x,y) = v_x \hat{i} + v_y \hat{j}
\]

\begin{tcolorbox}[title = Problem]	
  
  \[
  V_x, V_y = ? 
  \]
  
\end{tcolorbox}	

\begin{tcolorbox}[title = Model]

  1. Incompressible
  
   \[
     (a). \hspace{10px} 0 = \nabla \cdot \vec{v} = \frac{\partial v_x}{\partial x} + \frac{\partial
    v_y}{\partial y} 
  \] 
  2. Irrotational

  \[
    (b.) \hspace{10px} 0 = (\nabla \times \vec{v})_z = \frac{\partial v_x}{\partial x}
    - \frac{\partial v_x}{\partial y} 
  \]

\end{tcolorbox}

\begin{tcolorbox}[colback = red!5!white, colframe = red!50!black, title
  = Solving (a) and (b) ]
  
  Set of \textbf{coupled} partial differential equations (PDEs) 

  \begin{itemize}
    \item What are the Boundary Conditions? 
    \begin{itemize}
      \item an additional set of equations at the edges
    \end{itemize}
  \end{itemize}

  \begin{align*}
    &(1.) \hspace{10px} r = \sqrt{x^2 + y^2} \rightarrow \infty \hspace{10px}
    \vec{v} \rightarrow v_0 \hat{i} \\
    &(2.) \hspace{10px} \vec{v} \cdot \hat{r} = 0
  \end{align*}
  
 \textbf{Fact}: Complex Numbers \\
 Define $z = x + iy$, z is \textbf{not} the third coordinate\\
 Define $U = v_x \hat{i} - iv_y$ and $U = f(z)$ $\rightarrow$ Equations (a.)
 and
 (b.) are automatically satisfied. 

\end{tcolorbox}

\begin{tcolorbox}[colback = blue!5!white, colframe = blue!50!black, title
  = Solution]
  \[
  U = v_0\left(1-\frac{R^2}{z^2}\right)
  \]

  \[
  \frac{1}{z} = \frac{1}{x+iy} = \frac{x-iy}{x^2 + y^2}
  \]
  \[
  \frac{1}{z^2} = \frac{x^2 - y^2 - 2ixy}{(x^2 + y^2)^2}
  \]
  \[
  v_x = v_0 - \frac{v_0R^2(x^2 - y^2)}{(x^2 + y^2)^2}
  \]
  
\end{tcolorbox}

\paragraph{The Complex Plane}

%3
\begin{figure}[H]
  \centering
    \includegraphics[width = 10cm]{screenshot 55.png}
    \caption{complex plane}
\end{figure}

\begin{figure}[H]
  \centering
    \includegraphics[width = 7cm]{screenshot 56.png}
    \caption{quadrant's of complex plane in polar coordinates}
\end{figure}




\paragraph{Euler's Identity}

\[
  \cos \theta + i \sin \theta = e^{i\theta}
\]
\begin{align*}
  e^{x} &= 1 + \frac{x}{1} + \frac{x^2}{2} + \frac{x^3}{3!} + ...
  \sum_{n=0}^{\infty} \frac{x^n}{n!} \\
  e^{iy} &= 1 + \frac{iy}{1} - \frac{y^2}{2!} - \frac{iy^3}{3!} + \frac{y^4}{4!}
  + \dots \\
         &= \left(1 - \frac{y^2}{2!} + \frac{y^4}{4!} + \dots \right) + \left(\frac{y}{1}
- \frac{y^3}{3!} + \dots \right)i = \cos y + i \sin y 
\end{align*}

\begin{tcolorbox}[title = Euler's Identities]
  \begin{align*}
    e^{i\pi} &= -1 \\
    1 = e^{2\pi i} &= e^{2\pi n i} \hspace{10px} n = 0, \pm 1, \pm 2, \dots
\end{align*}

\end{tcolorbox}

\[
\log z = ?
\]
\begin{align*}
  z &= re^{i\theta} \\
  \log z &= \log r + i(\theta + 2 \pi n)
\end{align*}

\[
\sqrt{z}
\]
\begin{align*}
  \sqrt{re^{iz}} &= \sqrt{r} e^{i\theta / 2} \\
                 &= \sqrt{r}e^{\frac{i(\theta + 2\pi)}{2}}\\
                 &= -\sqrt{r} e^{i\theta / 2}
\end{align*}

\paragraph{Trigonometric Functions}

\[
  \cos z = \frac{e^{iz} + e^{-iz}}{2}
\]
\[
  \sin z = \frac{e^{iz} - e^{-iz}}{2i}
\]
\[
  cos(iy) = \frac{e^{-y} + e^{y}}{2} = \cosh y
\]
\[
  sin(iy) = i\frac{e^y - e^{-y}}{2} = i\sinh y
\]

\subsection{Hyperbolic Functions}

\[
\tanh = \frac{\sinh y}{\cosh y}
\]

Everything is \textbf{Real} from now on.

\paragraph{Identities} 

\begin{align*}
  \sinh (\alpha + \beta) &= \sinh \alpha \cosh \beta + \cosh \alpha + \sinh
  \beta \\
  \cosh (\alpha + \beta) &= \cosh \alpha \cosh \beta + \sinh \alpha + \sinh
  \beta \\
  \tanh (\alpha + \beta) &= \frac{\tanh \alpha + \tanh \beta}{1 + \tanh \alpha
  \tanh \beta}
\end{align*}

\paragraph{Applications to Special Relativity}

\textbf{Relativistic Addition to Velocities}

%4 
%A train moving, with a car moving inside of it, what would an observer
%calculate for the speed of the interior car? 

\begin{figure}[H]
  \centering
    \includegraphics[width = 6cm]{screenshot 57.png}
    \caption{a train moving with a car moving inside of it, what would an
    observer calculate for the speed of the interior car?}
\end{figure}



\[i
 W = \frac{u + v}{1 + \frac{uv}{c^2}} = c \frac{\tanh \alpha - \tanh\beta}{1
 + \tanh \alpha + \tanh \beta} = c\tanh(\alpha + \beta) 
\]
%5
%Rapidity - Using hyperbolic tanget establishes the bounds of velocity as c and
%-c

\begin{figure}[H]
  \centering
    \includegraphics[width = 8cm]{screenshot 58.png}
    \caption{rapidity - using hyperbolic tangent establishes the bounds of
    velocity as $c$ and $-c$}
\end{figure}





\end{document}  

