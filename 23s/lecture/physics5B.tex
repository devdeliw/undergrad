\documentclass[svgnames]{article}   	% use "amsart" instead of "article" for AMSLaTeX format
%\geometry{landscape}                	% Activate for rotated page geometry

%\usepackage[parfill]{parskip}    		% Activate to begin paragraphs with an empty line rather than an indent

\usepackage{graphicx}				          % Use pdf, png, jpg, or eps§ with pdflatex; use eps in DVI mode

%maths							                  % TeX will automatically convert eps --> pdf in pdflatex		
\usepackage{amssymb}
\usepackage{amsmath}
\usepackage{esint}
\usepackage{geometry}

%pgfplots
\usepackage{pgfplots}

%images
%\graphicspath{{ }}					          % Activate to set a image directory 

%tikz
\usepackage{pgfplots}
\pgfplotsset{compat=1.15}
\usepackage{comment}
\usetikzlibrary{arrows}
\usepackage[most]{tcolorbox}

%Figures
\usepackage{float}
\usepackage{caption}
\usepackage{lipsum}


\title{5B - Introductory Electromagnetism, Waves, and Optics}
\author{deval deliwala}
%\date{}							                % Activate to display a given date or no date

\begin{document}
\maketitle
%\section{}
%\subsection{}
\tableofcontents 					           % Activate to display a table of contents
\newpage

\noindent \textbf{Jan 17} 
\hrule
\section{Maxwell's Equations}

\begin{tcolorbox}	
  
    \[
      \nabla \cdot \vec{E} = \frac{\rho}{\varepsilon_0}         
    \]
    \[
      \nabla \times \vec{B} = \mu_0\vec{J} + \mu_0\varepsilon_0 \frac{\partial
      \vec{E}}{\partial t} 
    \]
    \[
      \nabla \times \vec{E} = - \frac{\partial \vec{B}}{\partial t} 
    \]
    \[
      \nabla \cdot \vec{B} = 0
    \]
\end{tcolorbox}	
\vspace{5px}

The goal of this course will be to understand Maxwell's Equations, and the
unison between the electric and magnetic field. 
\vspace{5px}

\begin{tcolorbox}[colback = red!5!white, colframe = red!50!black, title
  = Lorent'z Force eq]
  \[
  \vec{F} = q\vec{E} + q\vec{v} \times \vec{B}
  \]
\end{tcolorbox}
\vspace{5px}
\begin{tcolorbox}[colback = blue!5!white, colframe = blue!50!black, title
  = Rules]
  
  1. Charge in nature is quantized in units of $e$ \\
  2. Charge is conserved \\
  3. Charge has 2 types: $\pm$

\end{tcolorbox}
\vspace{5px}

$\vec{E}$ and $\vec{B}$ are \textit{vector fields}. This means $\vec{E}$ is
a function of every point in space: $\vec{E}(x,y,z)$

\vspace{5px}\[
  \vec{E}(r) = E_x(x,y,z)\hat{i} + E_y(x,y,z)\hat{j} + E_z(x,y,z)\hat{k}
\] \vspace{5px}
\newpage
$\nabla$ is a vector operator: 

\vspace{5px} \[
  \frac{\partial }{\partial x} \hat{i} + \frac{\partial }{\partial y} \hat{j}
  + \frac{\partial }{\partial x} \hat{k}
\] \vspace{5px}

\begin{tcolorbox}[colback = red!5!white, colframe = red!50!black, title
  = Vector Operations]
  
  \[
    \vec{A}\alpha \rightarrow \vec{B}
  \]
  \[
    \vec{A} \cdot \vec{B} \rightarrow \alpha
  \]
  \[
    \vec{A} \times \vec{B} \rightarrow \vec{C}
  \]

\end{tcolorbox}
\vspace{5px}

Consider a scalar field: $\varphi(\vec{r})$
\begin{align*}
&d\varphi = \varphi(r + dr) - \varphi(r) \\
&d\varphi = \frac{\partial \varphi(x,y,z)}{\partial x}dx + \frac{\partial
\varphi(x,y,z)}{\partial y}dy + \frac{\partial \varphi(x,y,z)}{\partial z}dz\\
&d\vec{r} = \hat{i}dx + \hat{y}dy + \hat{z}dz
\end{align*}
\begin{tcolorbox}
\[
  d\varphi = \nabla \varphi(\vec{r}) \cdot d\vec{r}
\]
\end{tcolorbox}
\vspace{5px}

$\nabla \varphi$ = "gradient of $\varphi$. " Also $\nabla \cdot \vec{E}
= \frac{\partial E_x}{\partial x}+ \frac{\partial E_y}{\partial y}
+ \frac{\partial E_z}{\partial z}$. This is known as the \textbf{divergence} of
$\varphi$. 
\vspace{10px}

The \textbf{curl} of $\varphi$ is equal to 

\vspace{5px} \[
\begin{vmatrix}
  \hat{i} & \hat{j} & \hat{k} \\
  P & Q & R \\
  \frac{\partial }{\partial x} & \frac{\partial }{\partial y} & \frac{\partial
  }{\partial z}  
\end{vmatrix}
\] \vspace{5px}
\[
  \rho = \text{charge density} = \frac{\text{number of particles} q}{dV} = nq
\] \vspace{5px}
where $n = \frac{\text{number of particles}}{dv}$


\vspace{5px} \[
  \vec{J} = \text{current density}
\] \vspace{5px}

\section{Statics}

\begin{tcolorbox}[colback = red!5!white, colframe = red!50!black, title
  = Equations of Electrostatics]
  
  \[
    \nabla \cdot \vec{E} = \frac{\rho}{\varepsilon_0}
  \]
  \[
    \nabla \times \vec{E} = 0
  \] 
  
\end{tcolorbox}

\begin{tcolorbox}[colback = blue!5!white, colframe = blue!50!black, title
  = Equations of Magnetostatics]
  
  \[
    \nabla \cdot \vec{B} = 0
  \]
  \[
    \nabla \times \vec{B} = \mu_0\vec{J}
  \] 

\end{tcolorbox}

\subsection{Flux}
\vspace{5px}
"Flux" = Flow 

\vspace{10px} 

Consider a fluid flow with a velocity vector field $\vec{v}(\vec{r})i$, flowing
into a small aperture defined by $\hat{n}\,d\vec{a}$, where  $\hat{n}$ is the
unit normal vector to the aperture, and $d\vec{a}$ is the area. 

\[
  d\Phi = \vec{v} \cdot d\vec{a}
\]

Relating to the Electric field, 

\[
  d\Phi_E = \vec{E}(\vec{r}) \cdot d\vec{a}
\] 
\[
  \int_S d\Phi_E = \Phi_E = \int_S \vec{E}(\vec{r}) \cdot d\vec{a}
\]\vspace{5px}

Green's, Gauss's, Divergence Theorem

\[
  \int_S \vec{E} \cdot d\vec{a} = \iiint_S \nabla \cdot \vec{E}(\vec{r}) \, d^3
  r
\]




\end{document}  

