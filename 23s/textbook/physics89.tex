\documentclass[svgnames]{article}   	% use "amsart" instead of "article" for AMSLaTeX format
%\geometry{landscape}                	% Activate for rotated page geometry

%\usepackage[parfill]{parskip}    		% Activate to begin paragraphs with an empty line rather than an indent

\usepackage{graphicx}				          % Use pdf, png, jpg, or eps§ with pdflatex; use eps in DVI mode

%maths							                  % TeX will automatically convert eps --> pdf in pdflatex		
\usepackage{amssymb}
\usepackage{amsmath}
\usepackage{esint}
\usepackage{geometry}
\usepackage{physics}

%pgfplots
\usepackage{pgfplots}

%images
%\graphicspath{{ }}					          % Activate to set a image directory 

%tikz
\usepackage{pgfplots}
\pgfplotsset{compat=1.15}
\usepackage{comment}
\usetikzlibrary{arrows}
\usepackage[most]{tcolorbox}

%Figures
\usepackage{float}
\usepackage{caption}
\usepackage{lipsum}




\title{Physics 89}
\author{deval deliwala}
%\date{}							                % Activate to display a given date or no date

\begin{document}
\maketitle

Adapted from \textit{Mathematical Methods in the Physical Sciences} by Mary L.
Boas 3rd ed. 

\tableofcontents 	                    % Activate to display a table of contents
\newpage

\section{Infinite Series, Power Series}
\vspace{5px}
\subsection{The Geometric Series}
\vspace{5px}

In a geometric progression we multiply each term by some fixed number to get
the next term. 

\begin{align*}
  &2,4,8,16,32,\dots,\\
  &1,\frac{2}{3}, \frac{4}{9}, \frac{8}{27}, \frac{16}{81}, \cdots, \\
  a, ar, ar^2, ar^3, \cdots 
\end{align*}
\vspace{5px}

Let us consider the expression 

\[
\frac{2}{3} + \frac{4}{9} + \frac{8}{27} + \cdots
\]

This expression is an example of an \textit{infinite series}, and we are asked
to find its sum. Let us first find the sum of $n$ terms, the formula being 

\vspace{5px}
\begin{tcolorbox}
\[
S_n = \frac{a(1-r^n)}{1-r}
\]
\end{tcolorbox}
\vspace{5px}
Using this equation, where $a$ is the first term and $r$ is the multiplier, we find

\vspace{5px} \[
S_n = \frac{2}{3} + \frac{4}{9} + \cdots + \left(\frac{2}{3}\right)^n
= \frac{\frac{2}{3}[1-\left(\frac{2}{3}\right)^n]}{1-\frac{2}{3}} = 2 \left[1
  - \left(\frac{2}{3}\right)^n\right]
\] \vspace{5px}

As $n$ increases, $\frac{2}{3}^n$ decreases and approaches zero, thus the sum
of the infinite series is 2. 
\vspace{5px}
Infinite Geometric Sequences are known as \textit{geometric series} and can be
written in the form 

\vspace{5px} \[
  a + ar + ar^2 + \dots + ar^{n-1} + \dots
\] \vspace{5px}

The sum of the geometric series is thus 

\vspace{5px} \[
  S = \lim_{n\to\infty} S_n
\] \vspace{5px}

A geometric series only has a sum if $|r| < 1$, and in this case, the sum is
also equal to

\vspace{5px} 
\begin{tcolorbox}
  \[
S = \frac{a}{1-r}
\] 
\end{tcolorbox}\vspace{5px}

The series is then called \textit{convergent}.

\subsection{Definitions and Notations}
\vspace{5px}
We can write series in shorthand form using summation: 

\vspace{5px} \[
\sum_{n=1}^{\infty} n^2 = 1^2 + 2^2 + 3^2 + \dots  
\] \vspace{5px}

\subsection{Convergent and Divergent Series}
\vspace{5px}

Convergent series have a finite sum, divergent series do not. You can not apply
ordinary algebra to divergent series. 

\vspace{5px} \[
1 + \frac{1}{2} + \frac{1}{3} + \frac{1}{4} + \dots \hspace{10px} \text{is
divergent}
\]
\[
  1 - \frac{1}{2} + \frac{1}{3} - \frac{1}{4} + \dots \hspace{10px} \text{is
    convergent}
\] \vspace{5px}
Given that $\lim_{n\to\infty} S_n = S$, we make the following definitions 

\begin{tcolorbox}	
  
  a. If the partial sums $S_n$ of an infinite series tend to a limit $S$, the
  series is called \textit{convergent}, otherwise it is \textit{divergent} \\
  b. The limiting value $S$ is called the $sum$ \\
  c. The difference $R_n = S - S_n$ is called the \textit{remainder}

\end{tcolorbox}	

\subsection{The Preliminary Test for Convergence}
\vspace{5px}

If the terms of an infinite series do \textit{not} tend to zero, the series
diverges. If $\lim_{n\to\infty} a_n = 0$, we must test further. 

\subsection{Convergence Tests for Series of Positive Terms; Absolute
Convergence}
\vspace{5px}

Four useful tests exist for series whose terms are all positive. We could also
use these tests on the \textit{absolute value} of negative series to determine
whether a series is \textit{absolutely convergent}, which also means it
converges as well, but with a different sum. 

\begin{tcolorbox}[colback = red!5!white, colframe = red!50!black, title = The
  Comparison Test]
  
  This test has two parts (a) and (b). 

  (a) Let 
  \[
  m_1 + m_2 + m_3 + m_4 + \dots 
  \]
  
  be a series of positive terms which you know converges. Then the series you
  are testing, namely 
  \[
  a_1 + a_2 + a_3 + a_4 + \dots
  \]
  
  is absolutely convergent if $|a_n| \leq m_n$ (that is, if the absolute value
  of each term of the $a$ series is no larger than the corresponding term of
  the $m$ series). 
  \vspace{5px}
  (b) Let 
  \[
  d_1 + d_2 + d_3 + d_4 + \dots 
  \]
  
  be a series of positive terms which you know diverges. Then the series 
  \[
  |a_1| + |a_2| + |a_3| + |a_4| + \dots
  \]
  diverges if $|a_n| \geq d_n$ for all $n$ from some point on.\\\\
  \textbf{Example} \\
  
  Test $ \sum_{n=1}^{\infty} \frac{1}{n!} = 1 + \frac{1}{2} + \frac{1}{6}i
  + \frac{1}{24} + \dots$ for convergence. 

  \vspace{5px}

  As a comparison series, choose the geometric series 

  \[
  \sum_{n=1}^{\infty} \frac{1}{2^n} = \frac{1}{2} + \frac{1}{4} + \frac{1}{8}
  + \frac{1}{16} + \dots 
  \]
  
  We know that $ \sum_{n=1}^{\infty} \frac{1}{2^n}$ converges because the ratio
  is $\frac{1}{2} \leq 1$, and since every corresponding sequence in
  $ \sum_{n=1}^{\infty} \frac{1}{n!}$ is less than $\frac{1}{2^n}$, we know
  that the series $\frac{1}{n!}$ converges as well. 

\end{tcolorbox}
\vspace{5px}
\begin{tcolorbox}[colback = blue!5!white, colframe = blue!50!black, title = The
  Integral Test]
  
  We can use this test when the terms of the series are positive \textit{and}
  not increasing, when $a_{n+1} \leq a_n$. The test states that \\\\

  If $0 \leq a_{n+1} \leq a_n$ for $n > N$, then $\sum^\infty a_n$ converges if
  $\int^\infty a_n \, dn$ is finite and diverges if the integral is infinite.
  (integral is evaluated only at the upper limit). \\\\

  \textbf{Example} \\

  Test for convergence the harmonic series $ \sum_{n=1}^{\infty} \frac{1}{n}$ 

  \[
  1 + \frac{1}{2} + \frac{1}{3} + \frac{1}{4} + \dots
  \]
  
  Using the integral test, we evaluate 

  \[
  \int^\infty \frac{1}{n} \, dn = \ln n \big|^\infty = \infty
  \]
  
  Since the integral is infinite, the harmonic series diverges


\end{tcolorbox}




\end{document}
